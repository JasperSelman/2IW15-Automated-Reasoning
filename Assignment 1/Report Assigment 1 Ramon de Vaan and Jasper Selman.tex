\documentclass[12pt]{article}
\usepackage{a4wide}
\usepackage{latexsym}
\usepackage{amssymb}
\usepackage{epic}
\usepackage{graphicx}
\pagestyle{empty}
\newcommand{\tr}{\mbox{\sf true}}
\newcommand{\fa}{\mbox{\sf false}}
\newcommand{\bimp}{\leftrightarrow}


\begin{document}
\begin{center}
\section*{Report 1 Automated Reasoning 2IW15 }
\end{center}
\begin{center}
\begin{tabular}{c c}
Jasper Selman & Ramon de Vaan\\
0741516 & 0758873\\
email: {\tt j.w.m.selman@tue.nl} & email: {\tt r.d.vaan@student.tue.nl}
\end{tabular}
\end{center}

\vspace{8mm}

\subsection*{Problem 2: Twelve Jobs}

We have to find a way to run 12 jobs numbered 1 to 12 satisfying the following requirements:
\begin{itemize}
\item The running time of job $i$ is $i+5$ for $i = 1,2, \ldots ,12.$
\item All jobs run without interrupt.
\item Job 3 may only start if jobs 1 and 2 have been finished.
\item Job 5 may only start if jobs 3 and 4 have been finished.
\item Job 7 may only start if jobs 3, 4 and 6 have been finished.
\item Job 8 may not start earlier than job 5.
\item Job 9 may only start if jobs 5 and 8 have been finished.
\item Job 11 may only start if job 10 has been finished.
\item Job 12 may only start if jobs 9 and 11 have been finished.
\item Jobs 5, 7 and 10 require a special equipment of which only one copy is available, so no two of these jobs may run at the same time.
\end{itemize}
We have to find a solution for which the total running time is minimized. 

\vspace{8mm}

\subsection*{Solution:}
Its hard to generalize this problem, since we cannot generailze all the requirements. However it is possible to generalize the first two requirements, but since that has no use in this problem we will not do this. 


We generalize this problem to putting $n$ queens on an $n \times n$
chess board, for any $n \geq 1$, with the same restriction that no 
two share a colum, row or diagonal.

For doing so, we introduce $n^2$  
 boolean variables $p_{ij}$ for $i,j = 1,\ldots,n$, where for 
every $i,j = 1,\ldots,n$ the value of $p_{ij}$ will be true if and
only if a queen will be put on position $(i,j)$, that is, in the
$i$-th row and in the $j$-th column.

As we have to put exactly $n$ queens, and no two are allowed to be
on the same row, every row should contain at least one queen. For
row $i$ this is expressed by the formula
\[ \bigvee_{j=1}^n p_{ij}.\]
In a similar way every column should contain at least one queen;
for column $j$ this is expressed by the formula 
\[ \bigvee_{i=1}^n p_{ij}.\]
Next we express that every row contains at most one queen, that
is, for every two distinct positions $j,k$ it is not allowed that
both $p_{ij}$ and $p_{ik}$ are true. For row $i$ this is expressed
by the formula
\[ \bigwedge_{j,k:1 \leq j < k \leq n} \neg p_{ij} \vee \neg p_{ik}.\]
Similarly, every column should contain at most one queen;
for column $j$ this is expressed by the formula 
\[ \bigwedge_{i,k:1 \leq i < k \leq n} \neg p_{ij} \vee \neg p_{kj} \]
Finally, we consider the requirements on diagonals. Two positions
$(i,j)$ and $(k,m)$ are on the same diagonal in the one direction
if and only if $i+k = j+m$, and they are on the same diagonal in 
the other direction if and only if $i-k = j-m$. For every pair of
such positions it is not allowed that they are both occupied by a
queen, so we require
\[ \neg p_{ij} \vee \neg p_{km}.\]
The total formula now consists of the conjunction of all these
ingredients, that is, 
\[ \bigwedge_{i=1}^n (\bigvee_{j=1}^n p_{ij}) \;\; \wedge \]
\[ \bigwedge_{j=1}^n (\bigvee_{i=1}^n p_{ij}) \;\; \wedge \]
\[  \bigwedge_{i=1}^n (\bigwedge_{j,k:1 \leq j < k \leq n} \neg 
p_{ij} \vee \neg p_{ik}) \;\; \wedge \]
\[ \bigwedge_{j=1}^n ( \bigwedge_{i,k:1 \leq i < k \leq n} \neg p_{ij}
\vee \neg p_{kj}) \;\; \wedge \]
\[ \bigwedge_{i,k:1 \leq i < k \leq n} ( \bigwedge_{j,m: i+k = j+m \vee
i-k = j-m} \neg p_{ij} \vee \neg p_{km}) \]

This formula is easily expressed in SMT syntax, for instance, for
$n=8$ one can generate

{\footnotesize

{\tt;;Define the maximum running time for comparison

(define maxTime::int 59)


;;Define the number of jobs

(define nrJobs::int 12)


;;Define the id's for all the jobs

(define-type ids (subtype (i::int) (and (>= i 1) (<= i nrJobs))))



;;Define a job as a record of an id, the running time, the start time and the end time

;;The first line makes sure the first requirement is met for all the jobs

;;The second line makes sure the second requirement is met for all the jobs

(define-type job 

	(subtype (j::(record id::ids runningTime::nat startTime::nat endTime::nat))

		(and 

			(= (select j runningTime) (+ (select j id) 5))

			(= (select j endTime) (+ (select j startTime) (select j runningTime)))

		)

	)

)


;;Define a function for jobs $j_1$ and $j_2$ such that $j_1$ can only start after $j_2$ is finished

(define startAfterEnd::(-> job job bool) (lambda (j1::job j2::job) 

(<= (select j2 endTime) (select j1 startTime))))


;;Define a function for jobs $j_1$ and $j_2$ such that $j_1$ can only start if $j_2$ is started

(define startAfterStart::(-> job job bool) (lambda (j1::job j2::job) 

(<= (select j2 startTime) (select j1 startTime))))


;;Define a function for jobs $j_1$ and $j_2$ such that $j_1$ and $j_2$ can not run at the same time

(define notConcurrent::(-> job job bool) (lambda (j1::job j2::job) 

(or (startAfterEnd j1 j2) (startAfterEnd j2 j1))))


;;Define all the jobs with the associated id

(define j1::(subtype (j::job) (= (select j id) 1)))

(define j2::(subtype (j::job) (= (select j id) 2)))

(define j3::(subtype (j::job) (= (select j id) 3)))

(define j4::(subtype (j::job) (= (select j id) 4)))

(define j5::(subtype (j::job) (= (select j id) 5)))

(define j6::(subtype (j::job) (= (select j id) 6)))

(define j7::(subtype (j::job) (= (select j id) 7)))

(define j8::(subtype (j::job) (= (select j id) 8)))

(define j9::(subtype (j::job) (= (select j id) 9)))

(define j10::(subtype (j::job) (= (select j id) 10)))

(define j11::(subtype (j::job) (= (select j id) 11)))

(define j12::(subtype (j::job) (= (select j id) 12)))


;;Define a time variable (to keep track of the total runningtime)

(define time::nat)


;;Make sure that if there are two jobs with the same id, they have the same starttime

(assert (forall (j1::job j2::job) (=> (= (select j1 id) (select j2 id) 

 (= (select j1 startTime) (select j2 startTime)))))


;;The requirements regarding specific jobs

(assert (startAfterEnd j3 j1))

(assert (startAfterEnd j3 j2))

(assert (startAfterEnd j5 j3))

(assert (startAfterEnd j5 j4))

(assert (startAfterEnd j7 j3))

(assert (startAfterEnd j7 j4))

(assert (startAfterEnd j7 j6))


(assert (startAfterStart j8 j5))


(assert (startAfterEnd j9 j5))

(assert (startAfterEnd j9 j8))

(assert (startAfterEnd j11 j10))

(assert (startAfterEnd j12 j9))

(assert (startAfterEnd j12 j11))


(assert (notConcurrent j5 j7))

(assert (notConcurrent j5 j10))

(assert (notConcurrent j7 j10))


;;Check that all the end times of jobs are smaller or equal to maxTime

(assert (forall (j::job) (<= (select j endTime) maxTime)))

(assert (forall (j::job) (<= (select j endTime) time)))

(assert (exists (j::job) (= (select j endTime) time)))


(check)}
}

Applying {\tt yices -e Exercise1-3.smt2} yields the following result
within a fraction of a second:

{\footnotesize

{\tt unknown}

{\tt (= time 59)}

{\tt (= J1 startTime 0)}

{\tt (= J1 endTime 6)}

{\tt (= J1 Duration 6)}

{\tt (= J2 startTime 0)}

{\tt (= J2 endTime 7)}

{\tt (= J2 Duration 7)}

{\tt (= J3 startTime 7)}

{\tt (= J3 endTime 15)}

{\tt (= J3 Duration 8)}

$\cdots \cdots$ }.

If we only look at the starting times we see that J1 = 0, J2 = 0, J3 = 7, J4 = 0, J5 =15, J6 = 7, J7 = 47, J8 = 15, J9 = 28, J10 = 0, J11 = 16 and J12 =42.  Since we know that for all jobs $i$ the running time is equal to $i + 5$. These starting times meet the requirements stated at the top. The total running time for this is 59. Now if we run the program with maxTime $\leq$ 58 we get unsat and therefore the minimum total running time for these requirements is 59.

\vspace{3mm}

{\bf Remark:} 

Our formula contains some redundancy: the requirement that every
row contains exactly one queen implies that there are exactly $n$
queens in total. By expressing that every column contains at least
one queen, from this property one concludes that also every column
contains at most one queen. We chose for this redundancy for
keeping the symmetry in the solution, and also following the
general strategy that introducing redundancy is often good for
efficiency.

\vspace{3mm}

{\bf Generalization:} 

As we generalized our approach for $n$ rather than 8, it is
interesting to see what happens for other values of $n$. For $n
> 10$ we have to take care of the notation: if we keep the
notation then it is not clear whether {\tt p111} represents 
$p_{1,11}$ or $p_{11,1}$. This is solved by putting an extra 
symbol between the two numbers. 

For $n=3$ the resulting formula is unsatisfiable, showing that
there is no solution. For $n = 4,5,6,\ldots$ the formula is
satisfiable, by which is a solution of the problem is found.
Efficiency is not a problem: for $n = 60$ there are 3600
variables and the formula consists of over 350,000 clauses, but 
still {\tt yices} succeeds in finding a solution within a few
seconds.
\end{document}
